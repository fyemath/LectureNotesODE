% !TEX root = ../main.tex
\chapter{Integrating Factors and Applications}
\chapterdate{9/27--9/30}


\section{Integrating Factors for Non-exact Equations}

Suppose 
\begin{equation}
  M(x,y) \D x  + N(x,y)\D y  =0
  \label{eq:2.3.1}
\end{equation}
is not exact. How to solve the equation? If it is a separable or linear first order, we already know how to solve. One of the method is to use integrating factors.
It seems reasonable to multiplying a non-zero factor to both sides of Equation \eqref{eq:2.3.1} to get an exact equation.

\begin{definition}
  A function $\mu(x, y)$ is called an \dfn{integrating factor} of Equation \eqref{eq:2.3.1} if the following equation is exact after multiplying it to both sides of Equation \eqref{eq:2.3.1}
  \[{\color{red}\mu(x,y)}M(x,y) \D x + {\color{red}\mu(x,y)}N(x,y)\D y=0.\]
\end{definition}

\begin{remark}
  It is possible to lose or gain solutions when multiplying by an integrating factor. In general, when using integrating factors, you should check whether any solution
  to $\mu(x, y)=0$ is in fact a solution to the original differential equation \eqref{eq:2.3.1}.
\end{remark}

{\red Do integrating factors always exist?}

This question is hard to answer in general. However, if we assume Equation \eqref{eq:2.3.1} has a general solution $F$, then the integrating factor exists and must be the ratio function 
\[\mu(x, y)=\frac{\frac{\partial}{\partial x} F}{M}=\frac{\frac{\partial}{\partial y} F}{N}.\]

{\red How to construct integrating factors?}

By Theorem \ref{thm:exactness}, a nonzero function $\mu(x, y)$ is an integrating factor if and only if it satisfies the equation
\[\frac{\partial}{\partial y}\Big(\mu(x,y)M(x,y)\Big)=\frac{\partial}{\partial x}\Big(\mu(x,y)N(x,y)\Big),\]
or equivalently
\begin{equation}
  \frac{\partial}{\partial y}\mu M - \frac{\partial}{\partial x} \mu N=\left(\frac{\partial}{\partial x} N-\frac{\partial}{\partial y} M\right)\mu.
  \label{eq:2.3.2}
\end{equation}

Equation \ref{eq:2.3.2} in general is even harder than the original equation to solve. Nevertheless, there are a few situations where Equation \ref{eq:2.3.2} can be solved relatively easily.

Clearly, if either $\frac{\partial}{\partial x} \mu=0$ or $\frac{\partial}{\partial y} \mu=0$, or equivalently, $\mu(x, y)=q(y)$ or $\mu(x, y)=p(x)$, then the above Equation \ref{eq:2.3.2} can be solved easily. That leads to the following proposition.

\begin{proposition}[Constructing Integrating Factor]\label{prop:intfactor} Given a differential equation $ M(x,y) \D x +  N(x,y) \D y =0$, suppose that $M$, $N$, $\frac{\partial}{\partial x} M$ and $\frac{\partial}{\partial y} N$ are continuous. Then  
\begin{enumerate}[label={Type \Roman*:}, leftmargin=*, align=left]
  \item If \[\frac{1}{{\color{red}N(x,y)}}(\frac{\partial}{\partial_ M}-\frac{\partial}{\partial_ N})={\color{red}p(x)}\]
  is a function {\color{red}only in $x$}, then we take integrating factor as a function of $x$, \[\mu(x,y)=e^{\int {\color{red}p(x)}\D x};\]
  \item If \[\frac{1}{{\color{red}M(x,y)}}(\frac{\partial}{\partial_ M}-\frac{\partial}{\partial_ N})={\color{red}q(y)}\]
  is a function {\color{red} only in $y$}, then we take integrating factor as a function of $y$,
  \[\mu(x,y)=e^{\int {\color{blue}-}{\color{red}q(y)}\D y}.\]
\end{enumerate}
\end{proposition}

\begin{remark}
	There are also some not so-obvious-cases that integrating factors can be constructed.
  
  If $u(x, y)=f(w(x, y))$, where $f$ is a single variable function, then an integrating factor is
  \[\mu(x, y)=\int e^{\dfrac{\frac{\partial}{\partial y} M - \frac{\partial}{\partial x} N}{N\frac{\partial w}{\partial_x}-M\frac{\partial w}{\partial_y}\D w}}.\]

  If $M$ and $N$ are both homogeneous functions, then an integrating factor is
  \[\mu(x,y) = \frac{1}{x M(x,y) + y N(x,y)}.
  \]

  If $M=yp(xy)$, $N=xq(xy)$, and $p(xy)\ne q(xy)$, then an integrating factor is
  \[\mu(x, y)=\frac{1}{xM-yN}.\]

  For more information, see \href{https://www.cfm.brown.edu/people/dobrush/am33/Mathematica/ch2/intfactor.html}{the section on Integrating Factors by Vladimir Dobrushkin}.
\end{remark}	

\begin{example} 
	Consider the equation
	\[ (2x^2+yx )  \D x +  (2xy+ x^2 )\D y  =0. \]
	\begin{enumerate}
		\item  Show that the equation is not exact;
		\item Show that $\dfrac{\frac{\partial}{\partial y} M(x,y) - \frac{\partial}{\partial x} N(x,y)}{N(x,y)}$ depends only on $x$;
		\item Reduce the equation to an exact equation.
	\end{enumerate}
\end{example}
\begin{solution}
    \begin{enumerate}
      \item Because $M(x, y)= 2x^2 + xy$ and $N(x, y) = 2xy+x^2$. The partial derivatives are
      \[
      \begin{split}
      \frac{\partial}{\partial y} M = & x, \\
      \frac{\partial}{\partial x} N = &  2y+2x. 
      \end{split}
      \]
      Hence $\frac{\partial}{\partial y} M \neq \frac{\partial}{\partial x} N$, and
      the equation is not exact.
      \item 
      We compute
      \[
      \frac{\frac{\partial}{\partial y} M(x,y) - \frac{\partial}{\partial x} N(x,y)}{N(x,y)} = \frac{x -(2y+2x)}{2xy+x^2} = \frac{-x-2y}{x(2y+x)} = -\frac{1}{x},
      \]
      which depends only on $x$.
      \item By Proposition \ref{prop:intfactor}, 
      We find an integrating factor of type I, 
      \[
      \mu = e^{\int \frac{\frac{\partial}{\partial y} M(x,y) - \frac{\partial}{\partial x} N(x,y)}{N(x,y)} \D x}= e^{\int - \frac{1}{x}\D x} =e^{-\ln x } = \frac{1}{e^{\ln x}}=\frac{1}{x}.
      \]
        Multiplying the equation by the integrating factor $\mu$ yields
      \[ 
      \begin{aligned}
        \frac{1}{x} ((2x^2+yx )  \D x+ (2xy+ x^2 ) \D y ) =& 0 \\
        (2x+y)  \D x +  (2y+ x ) \D y =& 0. 
      \end{aligned}
      \]
      This equation is exact because
      \[\frac{\partial}{\partial y}(2x+y)=1=\frac{\partial}{\partial x}(2y+x).\]
    \end{enumerate}
\end{solution}

\begin{example}
	Consider the equation
	\[y \D x+ (y-x) \D y  =0. \]
	\begin{enumerate}
		\item  Show that the equation is not exact, and $\dfrac{\frac{\partial}{\partial y} M(x,y) - \frac{\partial}{\partial x} N(x,y)}{M(x,y)}$ depends only on $y$.
		\item Reduce the equation to an exact equation.
		\item Solve the equation.
	\end{enumerate}
\end{example}
\begin{solution}
	\begin{enumerate}
		\item Because $M(x, y)=y$, $N(x, y)=y-x$ and 
		\[
		\begin{split}
		\frac{\partial}{\partial y} M(x, y) =&1 \\
		\frac{\partial}{\partial x} N(x, y) =&-1 \\
		\end{split}
		\]
		The equation is not exact but
		\[ \frac{\frac{\partial}{\partial y} M(x,y) - \frac{\partial}{\partial x} N(x,y)}{M(x,y)}=  \frac{2}{y}\] 
		depends only on $y$.
		\item By Proposition \ref{prop:intfactor}, 
			we look for an integrating factor of type II,
		\[I= e^{\int {\color{red}-}\frac{\frac{\partial}{\partial y} M(x,y) - \frac{\partial}{\partial x} N(x,y)}{M(x,y)}\D y} = e^{{\color{red}-}\int\frac{2}{y}} \D x=e^{{\color{red}-}2\ln y}=(e^{\ln(x)})^{{\color{red}-}2}=y^{{\color{red}-}2}=\frac{1}{y^2}. \]
		Multiplying $\mu$ to the non-exact equation yields
		\[
      \begin{aligned}
        \frac{1}{y^2} (y  \D x + (y-x)\D y)=&0\\
        \frac{1}{y}  \D x+\frac{y-x}{y^2} \D y  =& 0
      \end{aligned}
      \]
		This is an exact equation because
		\[
		\begin{split}
		\frac{\partial}{\partial y}\left(\frac{1}{y}\right) =&-\frac{1}{y^2} \\
		\frac{\partial}{\partial x}\left(\frac{y-x}{y^2}\right)=&-\frac{1}{y^2} \\
		\end{split}
		\]
		\item We look for $F$ such that 
		\[\frac{\partial}{\partial x} F  ~\D x+\frac{\partial}{\partial y} F ~\D y=  \frac{1}{y} \D x+\frac{y-x}{y^2} \D y.\]
		Equivalently, 
		\[
		\begin{cases}
		\frac{\partial}{\partial x} F  = \frac{1}{y}\\
		\frac{\partial}{\partial y} F  = \frac{y-x}{y^2}
		\end{cases}
		\]
		Integrating the first equation with respect to $x$ implies
		\[  
		F = \int \frac{1}{y} \D x  =\frac{x}{y} + h(y).
		\]
		Since $F$ has to be a solution solves the second equation and
		\[
		\frac{\partial}{\partial y} F = \frac{\partial}{\partial y}\left(\frac{x}{y} + h(y)\right) = -\frac{x}{y^2} + h'(y),
		\]
		the function $h(y)$ must satisfy the equation
		\[-\frac{x}{y^2} + h'(y) =\frac{y-x}{y^2},\]
		or equivalently,
		\[h'(y) =\frac1y.\]
		Therefore, a solution is $h=\ln y$. Hence, 
		\[F=\frac{x}{y}+\ln y\]
		and the general solution is 
		\[F=  \frac{x}{y} + \ln y=c.\]
	\end{enumerate}
\end{solution}

\begin{exercise}
	Consider the equation
	\[ - y  \D x+ x \D y  =0. \]
	\begin{enumerate}
		\item  Show that the equation is not exact, and $\dfrac{\frac{\partial}{\partial y} M(x,y) - \frac{\partial}{\partial x} N(x,y)}{N(x,y)}$ depends only on $x$.
		\item Reduce the equation to an exact equation.
		\item Solve the equation.
	\end{enumerate}
\end{exercise}
\begin{exersol}
	\begin{enumerate}
		\item Because $M(x, y)=-y$, $N(x, y)=x$ and 
		\[
		\begin{split}
		\frac{\partial}{\partial y} M(x, y) =&-1 \\
		\frac{\partial}{\partial x} N(x, y) =&1 \\
		\end{split}
		\]
		The equation is not exact but
		\[ \frac{\frac{\partial}{\partial y} M(x,y) - \frac{\partial}{\partial x} N(x,y)}{N(x,y)}=  \frac{-2}{x}\] 
		depends only on $x$.
		\item By Proposition \ref{prop:intfactor}, 
			we look for an integrating factor of type I,
		\[  I= e^{\int \frac{\frac{\partial}{\partial y} M(x,y) - \frac{\partial}{\partial x} N(x,y)}{N(x,y)}\D x} = e^{\int  \frac{-2}{x}} \D x=e^{-2\ln x}=(e^{\ln(x)})^{-2}=x^{-2}=\frac{1}{x^2}. \]
		Multiplying the equation by $\mu$ implies
		\[ \frac{1}{x^2} (- y  \D x+x \D y ) = 0  \]
		which is 
		\[ - \frac{y}{x^2}  \D x+\frac{1}{x} \D y  = 0  \]
		This is an exact equation because
		\[
		\begin{split}
		\frac{\partial}{\partial y} ( - \frac{y}{x^2} ) =&-\frac{1}{x^2} \\
		\frac{\partial}{\partial x} (\frac{1}{x})=&-\frac{1}{x^2} \\
		\end{split}
		\]
		\item We look for $F$ such that 
		\[\frac{\partial}{\partial x} F  ~\D x+\frac{\partial}{\partial y} F ~\D y=  - \frac{y}{x^2}  \D x+\frac{1}{x} \D y, \]
		Equivalently, 
		\[
		\begin{cases}
		\frac{\partial}{\partial x} F  =& -\frac{y}{x^2}\\
		\frac{\partial}{\partial y} F  =& \frac{1}{x}
		\end{cases}
		\]
		Integrating the first equation with respect to $x$ implies
		\[  
		F = \int  -\frac{y}{x^2} \D x  =\frac{y}{x} + h(y)
		\]
		Since $F$ has to satisfy the second equation and 
		\[
		\frac{\partial}{\partial y} F = \frac{\partial}{\partial y} (\frac{y}{x} + h(y)) = \frac{1}{x} + h'(y)
		\]
		the function $h(y)$ satisfies the equation
		\[  \frac{1}{x} + h'(y) =\frac{1}{x} \]
		or 
		\[ h'(y) =0 \]
		and $h=0$. Hence, 
		\[
		F=  \frac{y}{x}
		\]
		and the general solution is 
		\[
		F=  \frac{y}{x}=c
		\]
		Equivalently,
		\[
		y   = cx.
		\]
	\end{enumerate}
\end{exersol}

\begin{exercise}
	Consider the equation
	\[ (x-y)  \D x +  x \D y =0. \]
	\begin{enumerate}
		\item  Show the equation is not exact but $\dfrac{\frac{\partial}{\partial y} M(x,y) - \frac{\partial}{\partial x} N(x,y)}{N(x,y)}$ depends only on $x$.
		\item Reduce the equation to an exact equation.
		\item Solve the equation
	\end{enumerate}
\end{exercise}
\begin{exersol}
	\begin{enumerate}
		\item $M=x-y$, $N=x$ and 
		\[
		\begin{split}
		\frac{\partial}{\partial y} M =&-1 \\
		\frac{\partial}{\partial x} N =&1 \\
		\end{split}
		\]
		The equation is not exact but
		\[ \frac{\frac{\partial}{\partial y} M(x,y) - \frac{\partial}{\partial x} N(x,y)}{N(x,y)}=  \frac{-2}{x}\] 
		depends only on $x$.
		\item By Proposition \ref{prop:intfactor}, 
			we look for an integrating factor of type I, 
		\[\mu = e^{\int \frac{\frac{\partial}{\partial y} M(x,y) - \frac{\partial}{\partial x} N(x,y)}{N(x,y)}\D x} = e^{\int  \frac{-2}{x}\D x} = \frac{1}{x^2}. \]
		Multiplying the equation by $\mu$ implies
		\[ \frac{1}{x^2} [(x- y)  \D x +  x \D y] = 0  \]
		which is 
		\[  \frac{x-y}{x^2}  \D x+  \frac{1}{x} \D y= 0  \]
		This is an exact equation because
		\[
		\begin{split}
		\frac{\partial}{\partial y} (  \frac{x-y}{x^2} ) =&\frac{-1}{x^2} \\
		\frac{\partial}{\partial x} (\frac{1}{x})=&\frac{-1}{x^2} \\
		\end{split}
		\]
		\item 
		We look for $F$ such that
		\[
		\begin{split}
		\frac{\partial}{\partial x} F =& \frac{x-y}{x^2}\\
		\frac{\partial}{\partial y} F  =& \frac{1}{x}
		\end{split}
		\]
		
		Integrating the first equation with respect to $x$ implies
		\[  
		F = \int  \frac{x-y}{x^2} \D x =\int  \frac{1}{x}-\frac{y}{x^2} \D x =\ln x+ \frac{y}{x} + h(y)
		\]
		Since $F$ has to satisfy the second equation and 
		\[
		\frac{\partial}{\partial y} F = \frac{\partial}{\partial y} (\ln x+ \frac{y}{x} + h(y)) = \frac{1}{x} + h'(y),
		\]
		the function $h(y)$ satisfies the equation
		\[\frac{1}{x} + h'(y) =\frac{1}{x} \]
		or 
		\[ h'(y) =0 \]
		and $h=0$. Hence, 
		\[
		F=\ln x + \frac{y}{x}
		\]
		and the general solution is 
		\[
		\ln x + \frac{y}{x} = c,
		\]
    or
    \[y=cx-x\ln x.\]
	\end{enumerate}
\end{exersol}

\section{Existence and Uniqueness*}

Solving differential equations can be very complicated. It is impossible to find useful formulas for the solutions of most differential equations. However, knowing the existence and uniqueness can helps us looking for solutions.

\begin{theorem}[Picard's Existence and Uniqueness]
\begin{enumerate}
  \item 
  If $f$ is a function continuous on an open rectangle $R: \{a < x < b, c < y < d \}$
  that contains $(x_0,y_0)$, then the initial value problem
  \[y'=f(x,y), \quad y(x_0)=y_0\]
  has at least one solution on some open subinterval of $(a,b)$ that contains $x_0.$
  \item If both $f$ and $f_y$ are continuous on $R$ then the initial value problem
  \[y'=f(x,y), \quad y(x_0)=y_0\]
  has a unique solution on some open subinterval of $(a,b)$ that contains $x_0$
\end{enumerate}
\end{theorem}

A key to the proof of this theorem is the fact that that a continuously differential function $y(x)$ is a solution of the differential equation if and only if it satisfies the following integral equation:
  \[y(x)=y_0+\int_{x_0}^xf(t, y(t))\D t.\]

To prove the theorem, the French mathematician \'{E}ile Picard used a sequence of approximations with $y_0(x)=y_0$ and
\[y_{n}(x)=y_0+\int_{x_0}^xf(t, y_{n-1}(t))\D t.\]
This approximation is known as Picard's method of successive approximations. An interactive demonstration can be found on GeoGebra: \href{https://www.geogebra.org/m/jtnkbu72}{Picard's Method of Successive Approximations}.

The theorem can be proved by showing that $y_n(x)$ converges uniformly to a function $y(x)$ which is a solution. We refer the reader to \autocite[Chapter 13]{Simmons2016} for a proof.

\begin{example}
  Consider the initial value problem
\[y' = 3y^{2/3}, \quad y(0) = 0.\]
Show that both $y=0$ and $y=x^3$ are solutions.

Does it contradict Picard's Existence and Uniqueness Theorem?
\end{example}
\begin{solution}
  It is clear that $y=0$ is a solution. The function $y=x^3$ is also a solution because $y(0)=0^3=0$ and 
  \[y'=3x^2=2(x^3)^{\frac23}=3y^{\frac23}.\]

  This example does not contradict the uniqueness of the Theorem. Because the partial derivative $f_y=2y^{-\frac13}$ is not continuous along the line $(x,0)$.
\end{solution}

\begin{example}
  Consider the initial value problem
\[y' = 3xy^{\frac13}, \quad y(x_0) = y_0.\]
\begin{enumerate}
  \item 
  For what points $(x_0,y_0)$ does Picard's Existence and Uniqueness Theorem imply that this initial value problem has a solution?
  \item For what points $(x_0,y_0)$ does Picard's Existence and Uniqueness Theorem imply that this initial value problem has a unique solution on some open interval that contains $x_0$?
\end{enumerate}
\end{example}
\begin{solution}
  Because $f(x, y)=3xy^{\frac13}$ is continuous for all points $(x, y)$. The theorem implies that the initial value problem has a solution.

  The partial derivative 
  \[f_y=\frac{\partial}{\partial y}\left(3xy^{\frac13}\right)=3x\frac{\partial}{\partial y}y^{\frac13}=xy^{-\frac23}\]
  is undefined when $y=0$. For any point $(x, y)$ such that $y\ne 0$, $f_y$ is continuous. Therefore, by the theorem, the initial value problem has a unique solution for $y\ne 0$.

  Indeed, when $y_0=0$, the initial value problem has a trivial solution $y=0$ and an implicit solution $y^{\frac23}=x^2-x_0^2$. If $y_0\ne 0$, only $y^{\frac23}=x^2-x_0^2$ is a solution.
\end{solution}

\begin{exercise}
  Find all $(x_0,y_0)$ for which Picard's Existence and Uniqueness Theorem implies that the initial value problem 
  \[y'=\frac xy,\quad y(x_0)=y_0\] 
  has 
  \begin{enumerate*}
    \item 
    a solution and
    \item a unique solution
  \end{enumerate*}
  on some open interval that contains $x_0$.
\end{exercise}

\begin{exersol}
  Since $f(x, y)=\frac xy$ is continuous only when $y\ne 0$. The theorem implies that the initial value problem has a solution when $y_0\ne 0$.

  The partial derivative 
  \[f_y=\frac{\partial}{\partial y}\left(\frac xy\right)=x\frac{\partial}{\partial y}\left(\frac1y\right)=-\frac{x}{y^2}\]
  is discontinuous again when $y=0$. Therefore, by the theorem, the initial value problem has a unique solution for $y\ne 0$.
\end{exersol}

\section{Applications of First Order Differential Equations}

Some examples of applications of differential equations are mentioned in the introduction. In this section, we will discuss a few more examples.

\subsection{Exponential Growth}

When modeling population growth, Malthus's exponential model is frequently used:
\[P'=rP,\]
where $r$ is the constant.

As a separable equation, the general solution of this model is
\[P(t)=ce^{rt}.\]
With an initial condition $P(t_0)=P_0$, the solution is
\[P(t)=P_0e^{r(t-t_0)}.\]

In this model, normally, $P>0$. Therefore, if $a>0$, then $P$ is increasing without upper bound. If $a<0$, then $P$ is decreasing with the lower bound 0.

\begin{example}
	A bacteria culture starts with $10$ bacteria and grows to $90$ bacteria after $2$ hour.  Assume it grows at a rate proportional to its size.
	\begin{enumerate}
		\item Express the population after $t$ hours as a function of $t$.
		\item What is the population after $9$ hours?
		\item How long it will take for the population to reach $2500$?
	\end{enumerate}
\end{example}

\begin{solution}
	\begin{enumerate}
		\item Since the growth rate is constantly proportional to its size, the population satisfies the exponential model
		\[P'(t)=rP(t).\]
		Solving it yields the general solution
    \[P(t)=e^{rt+c_1}=e^{c_1}e^{rt}=ce^{rt},\]
		where $c$ is a constant to be determined by an initial condition.
		
		In this model, both $c$ and $a$ are to be determined. In the statement of the question, the sentence ``A bacteria culture starts with 10 bacteria and grows to $90$ bacteria after $2$ hour'' means 
		\[P(0)=10\quad\textup{ and}\quad P(2)=90.\]
  Those two conditions implies that $a$ and $c$ satisfy the following system of equations
		\[
			\begin{cases}
				& N(0)=c\cdot e^0=10\\[0.5em]
				& N(2)=c\cdot e^{2a}=90
      \end{cases}
		\]
    The first equation implies that $c=10$. Plugging it in to second equation yields
    \[
      \begin{aligned}
        e^{2a}=&9\\
        2a=&\ln9\\
        a=&\frac12\ln9\\
        a=&\ln3.
      \end{aligned}
    \]
  Hence the population function is 
  \[P(t)=10e^{t\ln3}.\]

		\item  The population after $9$ hours is
		\[P(9)=10e^{9\ln 3}\approx 196830.\]
		\item The time that it will takes the culture to $2500$ satisfies the equation \[N(t)=2500.\]
		Solving the equation yields
    \[
      \begin{aligned}
        10e^{t\ln3}=&2500\\
        e^{t\ln3}=&250\\
        t\ln3=&\ln(250)\\
        t=&\frac{\ln(250)}{\ln 3}\\
        t\approx & 5.
      \end{aligned}
    \]

    So it takes about $5$ hours for the bacteria culture to grow to $2500$.
	\end{enumerate}
\end{solution}

In the previous population model, when $a>0$, the population grows exponentially without a limit. In reality, the growth is limited by the environment capacity $\frac{1}{\alpha}$. A refined model is Verhulst's logistic population model, 
\[P'=rP(1-\alpha P),\]
where $a$ is the growth rate when the capacity has no or minimal impact on the growth, and $\frac1{\alpha}$ is capacity, i.e. the limit of the population in the environment.

Note that the equation is also an separable (indeed, autonomous) equation. The equation can be solved using partial fraction decomposition:
\[
\begin{aligned}
  P'=&rP(1-\alpha P)\\
  \frac{P'}{P(1-\alpha P)}=&r\\
  \frac{P'}{P}+\frac{\alpha P'}{1-\alpha P}=&r\\
  (\ln(P))'-(\ln(1-\alpha P))'=&r\\
  \left(\ln\left(\frac{P}{1-\alpha P}\right)\right)'=&r\\
  \frac{P}{1-\alpha P}=&e^{rt}\\
  P=&e^{rt}(1-\alpha P)\\
  P+e^{rt}\alpha P=&ce^{rt}\\
  P=&\frac{ce^{rt}}{1+c\alpha e^{rt}}.
\end{aligned}  
\]

Note that the limit of $P(t)$ as $t\to \infty$ is nothing but the capacity $\frac1\alpha$. 

\begin{example}
	One hundred rabbits were released in a forest. . It is observed that the population after $t$ years develops according to Verhulst's logistic population model. The carrying capacity is estimated to be $10,000$.
  
  Suppose the growth rate is $r=2$. What is the population size after 5 years.
\end{example}
\begin{solution}
	Since the carrying capacity is $\frac1\alpha=10,000$, the value $\alpha$ is $0.00001$.  
	The logistic model is then
  \[P'(t)=2P(t)(1-0.00001P(t)).\]
	The discussion of the logistic model above shows that the general solution is
  \[P(t)=\frac{ce^{2t}}{1+0.00001ce^{2t}}.\]
  Since $P(0)=100$, solving for $c$ yields $c\approx 100$.

  So the population of rabbits after $t$ years is
  \[P(t)=\frac{100e^{2t}}{1+0.001e^{2t}}.\]

  Therefore, after 10 years the population will be
  \[P(15)=\frac{100e^{10}}{1+0.001e^{10}}\approx 95657.\]
\end{solution}	

\begin{exercise}
  A bread dough increases in volume at a rate proportional to the volume $V$ present. Suppose the initial volume is $V_0$. After 2 hours, the volume increases to $1.5V_0$. How long will it take for the volume to increase to $2V_0$?
\end{exercise}
\begin{exersol}
  Suppose the proportional factor is $k$. Then $V'(t)=kV(t)$ after $t$ hours. Since the initial condition is $V(0)=V_0$, it follows that $V(t)=V_0e^{-kt}$.
  
  Because $V(2)=1.5V_0$. The constant $k$ satisfies the equation
  \[1.5V_0=V_0e^{2k}.\]
  Solving the equation yields $k=\frac12\ln(1.5)$.

  The time needed for the volume to $2V_0$ satisfies the equation
  \[V_0e^{t\frac{\ln(1.5)}{2}}=2V_0.\]
  Solving for $t$ from the equation gives \[t=\frac{2\ln2}{\ln(1.5)}\approx 3.4.\]
\end{exersol}

\subsection{Exponential decay}

If a quantity decreases at a rate proportional to its current value, then we say it is subject to exponential decay. Suppose the quantity is $N(t)$ after a time $t$ and the rate of decreasing, called the \emph{exponential decay rate}, is $k>0$. Then the quantity $N(t)$ satisfies the following equation
\[{\frac {dN}{dt}}=-k N.\]

Solving the equation yields that
\[N(t)=N_0\cdot e^{-kt},\]
where $N_0=N(0)$ is the initial quantity.

Exponential decay applies in a wide variety of situations, particularly in natural science. For example, the quantity of radioactive material decays exponentially.

In a radioactive decay model, the decay rate $k$ can be determined by the \emph{half-life}, that is the time required to decay the quantity to one half of its initial value. Suppose the half-life of a radioactive material is $\tau$. Then the exponential decay rate $k$ satisfies the following equation
\[N_0e^{-k\tau}=\frac12N_0.\]
Solving for $k$ implies that
\[k=\frac{\ln2}{\tau}\]
and the exponential decay model becomes
\[N(t)=N_0e^{-\ln 2\cdot\frac{t}{\tau}}=N_02^{-\frac{t}{\tau}}.\]

\begin{example}
  A radioactive substance has a half-life of 40 days. Suppose its mass is now 300 g (grams).
  
  After how long will the amount present be 200 g.
\end{example}
\begin{solution}
  Applying the radioactive decay model with $\tau=40$ and $N_0=300$ implies that
  \[N(t)=300\cdot 2^{\frac{t}{40}}.\]
  When $N(t)=200$, the time $t$ satisfies the equation
  \[200=300\cdot 2^{-\frac{t}{40}}.\]
  Solving the equation for $t$ yields
  \[\begin{aligned}
    200=&300\cdot 2^{-\frac{t}{40}}\\
    \frac{2}{3}=&2^{-\frac{t}{40}}\\
    \ln\left(\frac{2}{3}\right)=&\ln2\cdot\left(-\frac{t}{40}\right)\\
    t=&\frac{40(\ln3-\ln2)}{\ln2}\\
    t\approx & 23.
  \end{aligned}\]
  So it takes about 23 days for the mass decrease to 200 g.
\end{solution}

\begin{example}[Mixed Growth and Decay]
  A radioactive substance has a half-life of 100 days. Suppose its mass is now 500 mg and additional amounts are added at the rate of 4 mg per day. Suppose the mass after $t$ days is $Q(t)$.
  
  \begin{enumerate}
    \item 
    Find a formula for the rate of change $Q'(t)$ of the mass in terms of $t$.
    \item Find $Q(t)$ in terms of $t$.
  \end{enumerate}
\end{example}
\begin{solution}
  \begin{enumerate}
    \item 
    The rate of change composes of two rates: the rate of decreasing and rate of increasing. 
    
    By exponential decay model, the The rate of decreasing is $-kQ(t)$, where $k=\frac{\ln 2}{100}$ is the exponential decay rate determined by the half-life.
  
    The rate of increasing is the fixed rate 5 mg/day.
  
    Therefore, the rate of change is
    \[Q'(t)=-\frac{\ln(2)}{100}Q(t)+5.\]
  
    \item 
    Since the right hand side of the above equation is a linear expression of $Q(t)$, a linear substitution $v=-\frac{\ln2}{100}Q(t)+5$ will reduce the differential equation to the separable equation
    \[v'=-\frac{\ln2}{100}v.\]
    Therefore,
    \[v=ce^{-\frac{\ln2}{100}t}\]
    and
    \[Q(t)=-\frac{100}{\ln2}\left(ce^{-\frac{\ln2}{100}t}-5\right).\]

    The initial condition $Q(0)=500$ implies
    \[
    \begin{aligned}
      -\frac{100}{\ln2}\left(c-5\right)=&500\\
      \left(c-5\right)=&-5\ln 2\\
      c=&-5\ln 2+5.
    \end{aligned}  
    \]
    Therefore, 
    \[Q(t)=\frac{(500\ln2-500)2^{-\frac{t}{100}}+500}{\ln2}.\]
  \end{enumerate}
\end{solution}

\begin{exercise}
   Living cells maintain a consistent level of carbon-14. However, when the cell dies, carbon-14 start decaying exponentially at a constant rate. It is known that the half-life of carbon-14 is about 5570 years. 
   
   An archaeologist investigating the site of an ancient village finds a burial ground where the amount of carbon-14 present in individual remains is about 63\% of the amount present in live individuals. Estimate the age of the village.
\end{exercise}
\begin{exersol}
Let $Q=Q(t)$ be the quantity of carbon-14 remained in individuals $t$ years after death, and let $Q_0$ be the quantity that would be present in live individuals. Since carbon-14 decays exponentially with half-life 5570 years, its exponential decay rate is
\[k=\frac{\ln2}{5570}.\]
Therefore,
\[Q=Q_0e^{-t(\ln2)/5570}.\]

Since the quantity remained in individuals now is $Q(t)=0.63Q_0$, the age $t$ satisfies the equation
\[Q_0e^{-t(\ln2)/5570}=0.63Q_0\]
Solving the equation for $t$ yields
\[t=-5570 \frac{\ln(0.63)}{\ln2} \approx 3713.\]

Therefore, the village is about 3713 years old.
\end{exersol}

% Suppose an amount of money $Q_0$ is deposited in a savings account and there will be no further deposits or withdrawals for $t$ years, during which the account bears the interest at a constant annual rate $r$.

% If the interest is compounded $n$ times per year, the value of the account is multiplied $n$ times per year by $(1+r/n)$; therefore, the value of the account after $t$ years is determined by the following multiple compounding model
% \[Q(t)=Q_0\left(1+{r}{n}\right)^{nt}.\]

% If the interest is compounded continuously, which means that $n\to\infty$, then
% \[\lim_{n\to\infty} \left(1+{r}{n}\right)^n=e^r,\]
% which implies the continuous compounding model
% \[
%   \begin{aligned}
%     Q(t) & =\lim_{n\to\infty} Q_0\left(1+{r}{n}\right)^{nt}=Q_0 \left[ \lim_{n\to\infty} \left(1+{r}{n}\right)^n\right]^t \\
%      &=Q_0e^{rt}. 
%   \end{aligned}
% \]
% Therefore, with continuous compounding the value of the account grows exponentially, and $Q(t)$ is a solution to the initial value problem
% \[Q'(t)=rQ(t),\qquad Q(0)=Q_0.\]

\subsection{Newton’s law of cooling}

Newton’s law of cooling states that the temperature of an object changes at a rate proportional to the difference between its temperature and the temperature of its surrounding.

Let $T(t)$ be the temperature of the object and $T_m(t)$ the temperature of the surrounding medium at the time $t$. Then Newton's law of cooling can be state as a differential equation:
\[T'(t)=-k(T(t)-T_m(t)).\]
Here $k$ is a positive constant, called the temperature decay constant. The reason that $k$ is positive is because the temperature of the object must decrease if $T(t) > T_m(t)$, or increase if $T(t) < T_m(t)$.

For simplicity, assume that the medium is maintained at a constant temperature $T_m$. This model apply to many situations but not all. For example, to cool down a cup of hot coffee in the room temperature, the change of room temperature is neglectful. However, to cool down a cup of hot coffee is a small pot of cold water, the temperature of the water will change accordingly. In the later case, the heat transfer law in thermodynamics will be needed.

In this section, the surrounding temperature will be assumed to be constant. In this case, note that $T'(t)=(T(t)-T_m)'$, then Newton's law of cooling can be re-formulated as 
\[(T(t)-T_m)'=-k(T(t)-T_m).\]
That is, $T-T_m$ decays exponentially, with decay constant $k$. Therefore,
\[T(t)-T_m=(T_0-T_m)e^{-kt}\]
or equivalently
\[T(t)=(T_0-T_m)e^{-kt}+T_m,\]
where $T_0$ is the initial temperature of the object.

\begin{example}
  A ceramic insulator is baked at $400^\circ$C and cooled in a room in which the temperature is $25^\circ$C. After 4 minutes the temperature of the insulator is $200^\circ$C. What is its temperature after 8 minutes?
\end{example}
\begin{solution}
  Here $T_0=400$ and $T_m=25$, so the temperature function of the ceramic insulator is
  \[T=25+375e^{-kt}.\]
  
  Since $T(4)=200$, which determines $k$, then
  \[200=25+375e^{-4k}.\]

  Solving for $k$ yields
  \[k=-\frac{1}{4}\ln\left(\frac{7}{15}\right).\]
  
  Substituting this into $T(t)$ yields
  \[T=25+375 e^{-\frac{t}{4}\ln\left(\frac{7}{15}\right)}.\]

  Therefore, the temperature of the insulator after 8 minutes is
  \[
    \begin{aligned}
      T(8) & = 25+375 e^{-2\ln\left(\frac{7}{15}\right)} \\
      & = 25+375 \left(\frac{7}{15}\right)^2 \approx 107^\circ \textup{ C}. 
    \end{aligned}
  \]
\end{solution}

\begin{exercise}
	A metal bar at a temperature of $200^\circ$F is placed in a room at a constant temperature of $50^\circ$F. If after $20$ minutes, the temperature of the bar is $90^\circ$F, find 
	\begin{enumerate}
		\item Find the formula of temperature as a function of $t$. 
		\item Find the time it will take the bar to reach a temperature of $60^\circ$F;
		\item Find the limit of the temperature as time goes to infinity. 
	\end{enumerate}	
\end{exercise}
\begin{exersol}
  \begin{enumerate}
    \item Since $T_m=50$ and $T_0=200$, the temperature function is
    \[T(t)=50+150e^{-kt}.\]
    Because $T(20)=90$, then $k$ is determined by the equation
    \[90=50+ 150\cdot e^{-20k}.\]
    Solving the equation yields
    \[
    \begin{aligned}
    40=&150\cdot e^{-20k}\\
    e^{-20k}=&\frac{4}{15}\\
    -20k=&\ln\left(\frac{4}{15}\right)\\
    k=&-\frac{1}{20}\ln\left(\frac{4}{15}\right).
    \end{aligned}  
    \]
    Hence 
    \[T(t)=50+150e^{\frac{1}{20}\ln(\frac{4}{15})t}.\] 

  \item   
    The time that it will take the bar to $60^\circ$ satisfies the equation
    \[50+150\cdot e^{-kt}=60,\]
    where $k=-\frac{1}{20}\ln\left(\frac{4}{15}\right)$.

    Solving the equation yields
    \[
    \begin{aligned}
      150\cdot e^{-kt}=&10\\
      e^{-kt}=&\frac{1}{15}\\
      -kt=&\ln(\frac{1}{15})\\
      t=&-\frac{1}{k}\ln(\frac{1}{15})\\
      t=& -\frac{1}{-\frac{1}{20}\ln(\frac{4}{15})}\ln(\frac{1}{15})\\
      t=&20\cdot \frac{\ln(1/15)}{\ln(4/15)}\\
      t\approx&41.
    \end{aligned}  
    \]
    So it takes about 41 minutes for the bar to cool down to $60^\circ$F.

    \item When $t$ goes to infinity, because $k$ is positive, the limit $\lim\limits_{t\to\infty}e^{-kt}=0$. 
    Hence 
    \[\lim\limits_{t\to \infty}T(t)=50+150\cdot \lim\limits_{t\to \infty} e^{\frac{1}{20}\ln(\frac{4}{15})t}=50.\]
    The answer is the room temperature.
  \end{enumerate}   
\end{exersol}

\subsubsection{Mixing Problems}

\begin{example}
  A tank initially contains 40 pounds of salt dissolved in 600 gallons of water. Starting at $t_0 = 0$, water that contains 1/2 pound of salt per gallon is poured into the tank at the rate of 4 gal/min and the mixture is drained from the tank at the same rate.
  \begin{enumerate}
    \item Find a differential equation for the quantity $Q(t)$ of salt in the tank at time $t > 0$, and solve the equation to determine $Q(t)$.
    \item Find $\lim\limits_{t\to\infty}Q(t)$.
  \end{enumerate}
\end{example}
\begin{solution}
To find a differential equation for the quantity of salt $Q$, because the given information is about the rate of change of the quantity of salt $Q'$, we will find an equation for $Q'$. is the rate of change of the quantity of salt in the tank changes with respect to time; The rate of change composes of two parts
\[Q' = \textup{ in-rate}-\textup{ out-rate}.\]

The in-rate is
\[\left(\frac{1}{2}\ \textup{ lb/gal}\right) \times (4\ \textup{ gal/min}) = 2\ \textup{ lb/min}.\]

Since the concentration changes, the determine the out-rate, we need know the concentration at $t$. Because the in-flow and out-flow rates are the same, the volume of the mixture is constant which is $600$ gal. Therefore, the concentration at time $t$ is $\frac{Q(t)}{600}$, and the out-rate is then
\[(\textup{ concentration})\times(\textup{ rate of flow out})=\frac{Q(t)}{00}\times 4=\frac{Q(t)}{150}.\]

Therefore,
\[Q' = 2-\frac{Q}{150},\]
which is a first order linear equation that can be rewritten as
\[Q'+\frac{Q}{150} = 2.\]
The integrating factor is
\[r(t)=e^{\int\frac{1}{150}\D t}=e^{\frac{t}{150}}.\]
Multiplying the integrating factor to both sides and integrating both sides yields 
\[\begin{aligned}
  Q(t)e^{\frac{t}{150}}=&\int 2e^{\frac{t}{150}}\D t\\
  Q(t)e^{\frac{t}{150}}=&300e^{\frac{t}{150}}+c\\
  Q(t)=&300+ce^{-\frac{t}{150}}.
\end{aligned}
\]
Since $Q(0)=40$, $c=-260$ and
\[Q=300-260e^{-t/150}.\]

The limit of $Q(t)$ as $t\to\infty$ is
\[\lim_{t \to \infty}Q(t)=\lim\limits_{t\to \infty}\left(300-260e^{-t/150}\right)=300.\]
This is intuitively reasonable. Because, the incoming solution contains 1/2 pound of salt per gallon and there are always 600 gallons of water in the tank.
\end{solution}

\begin{example}
  A 500-liter tank initially contains 10 g of salt dissolved in 200 liters of water. Starting at $t_0=0$, water that contains 1/4 g of salt per liter is poured into the tank at the rate of 4 liters/min and the mixture is drained from the tank at the rate of 2 liters/min. Find a differential equation for the quantity $Q(t)$ of salt in the tank at time $t$ prior to the time when the tank overflows and find the concentration $K(t)$ (g/liter) of salt in the tank at any such time.
\end{example}
\begin{solution}
  The difference between this example and the above equation is that the volume of the mixture changes in this example. Let $W(t)$ of solution in the tank at any time $t$ prior to overflow. Since $W(0) = 200$, in-flow rate 4 liters/min and the out-flow rate 2 liters/min, the net flow rate is then 2 liters/min. Therefore, the volume is
  \[W(t) = 200+\int_0^t 2\D x=200+2t.\]
  
  Since the volume of tank is 500 liter, and $W(150)=500$, this formula is valid for when the time $t$ is from $0$ to $150$ min, i.e. $0\le t\le 150$.
  
  Now let $Q(t)$ be the number of grams of salt in the tank at time $t$, where $0 \le t \le 150$. As in above example  
  \[
    \begin{aligned}
      Q' =& \textup{ rate in}-\textup{ rate out}\\
        =&\left(\frac{1}{4} \textup{ g/liter}\right) \times (4 \textup{ liters/min}) + Q(t)\cdot\frac{2}{2t+200} \textup{ g/min}\\
        =&1\textup{ g/min} + Q(t)\cdot \frac{1}{t+100} \textup{ g/min}\\
        =&\left(1 + \frac{Q(t)}{t+100}\right) \textup{ g/min}.
    \end{aligned}
    \]
  Therefore, the salt flow rate is
  \[Q'=1-\frac{Q}{t+100},\]
  or
  \[Q'+\frac{1}{t+100}Q=1.\]
  The integrating factor is
  \[r(x)=e^{\int \frac{1}{t+100}\D t}=t+100.\]
  Multiplying the equation by $r(x)$ and integrating both sides yields
  \[Q(t)=\frac{1}{t+100}\int(t+100)\D t=\frac{t+100}{2}+\frac{c}{t+100}.\]
  Since $Q(0)=10$, solving for $c$ implies
  \[c=-4000.\]
  Hence,
  \[Q(t) =\frac{t+100}{2}-\frac{4000}{t+100}.\]
  
  Now let $K(t)$ be the concentration of salt at time $t$. Then
  \[K(t)=\frac{Q(t)}{W(t)}= \frac{1}{4}-\frac{2000}{(t+100)^2}.\]
\end{solution}


\subsection{Motion Under Gravity in a Resisting Medium*}

In this section, we will consider an object with constant mass $m$ moving vertically in a resisting medium near Earth's surface under a force $F(t)$. We will take the {\color{red}upward} direction to be {\color{red}{positive}}. Let $y=y(t)$ be the displacement of the object from a reference point above the ground at time $t$. Let $v=v(t)$ and $a=a(t)$ be the velocity and acceleration of the object at time $t$. Then $a(t)=v'(t)$ and $v(t)=y'(t)$.

Newton’s second law of motion asserts that the force $F$ equals the product of the mass $m$ and the acceleration $a$:
\[F(t)=ma(t).\]

When an object moves vertically in a resisting medium, two main forces are the gravitational force $-mg$ and the medium resistive force. Here, $g$ is the acceleration due to gravity.

\begin{example}
  An object with mass $m$ and initial velocity $v(0)=v_0$ moves under constant gravitational force $mg$ through a medium that exerts a resistance with magnitude proportional to the speed $|v|$ of the object.
  \begin{enumerate}
    \item Find the velocity of the object as a function of $t$.
    \item Find the limit $\lim\limits_{t\to\infty}v(t)$.
  \end{enumerate}
\end{example}

\begin{solution}

Taking the upward as the positive direction, the total force acting on the object is
\[F=-mg+F_1,\]
where $-mg$ is the force due to gravity and $F_1$ is the resisting force of the medium. Since, the resisting force $F_1$ has the magnitude proportional to $|v|$. There is a positive constant $k$ such that $|F_1|=mk|v|$. Because the resistance force is always opposite the direction of the velocity.
If the object is moving downward, that is $v\le 0$, the resisting force is upward. So 
\[F_1=mk(-v)=-mkv>0.\]
If the object is moving upward, that is $v\ge 0$, the resisting force is downward. So
\[F_1=-mkv<0\]
Therefore, the total force $F$ is
\[F=-mg-mkv,\]
regardless of the sign of the velocity.

From Newton’s second law of motion,
\[F=ma=mv'.\]
So
\[mv'=-mg-mkv,\]
or equivalently,
\[v'=-kv-g.\]
This equation can be solved using multiple methods. Because the coefficients are constants, we may use the linear substitution $v=y-\frac{g}{k}$. Plugging it into the above equation yields
\[y'=-ky,\]
which has the general solution $y=ce^{-kt}$. Hence,
\[v=ce^{-kt}-\frac{g}{k}.\]
Since $v(0)=v_0$, the constant $c$ is determined by
\[c-\frac{g}{k}=v_0.\]
Therefore, $c=v_0+\frac{g}{k}$.

So the velocity function is
\[v=\left(v_0+\frac{g}{k}\right)e^{-kt}-\frac{g}{k}\]

Letting $t\to\infty$ here shows that the limit of $v(t)$ is
\[\lim_{t\to\infty} v(t)=-\frac{g}{k}.\]
\end{solution}

We see that under reasonable assumptions on the resisting force, the velocity approaches a limit as $t\to\infty$, which is called the \emph{terminal velocity}. You will find an object reaches its terminal velocity when its acceleration is $0$.

\begin{example}
  A 10 kg mass is given an initial velocity $v(0)=v_0\le 0$ near Earth’s surface. The only forces acting on it are gravity and atmospheric resistance proportional to the square of the speed. Assuming that the resistance is 8 Newton (N for short) when the speed is 2 meters/second (m/s for short).
  \begin{enumerate}
    \item Find the velocity of the object as a function of $t$.
    \item Find the terminal velocity.
  \end{enumerate}
\end{example}
\begin{solution}
Since the object is falling, the resistance is in the upward direction which is assumed to be the positive direction. Hence,
\[mv'=-mg+kv^2,\]
where $k$ is a constant. Since the magnitude of the resistance is 8 N when $v=2$ m/s, that is,
\[k(2^2)=8.\]
So $k=2\ \textup{ N-s}^2/\textup{ m}^2$, where $N-s$ means the product of $N$ and $s$. Since $m=10$ kg and $g=9.8 ~ \textup{ m}/\textup{ s}^2$, the velocity satisfies the following equation.
\[10v'=-98+2v^2=2(v^2-49).\]

Note that the solutions $v=-7$ or $v=7$ of the equation $v^2-49=0$ are solutions to the above differential equation. Since $v_0\le 0$ and the object is falling, the velocity $v$ has a negative direction. So $v=7$ can not be a solution under the assumption that $v_0\le 0$. Moreover, $v=-7$ is a solution only when $v_0=-7$.

Now assume $v_0\ne-7$. Separate variables implies
\[\left(\frac{1}{v^2-49}\right)v'=\frac{1}{5}.\]
Integrating both sides using partial fraction decomposition yields
\[
\begin{aligned}
  \left(\frac{1}{v^2-49}\right)v'&=\frac15\\
  \left(\frac{1}{(v-7)(v+7)}\right)=&\frac15\\
  \frac{1}{14}\left(\frac{1}{v-7} - \frac{1}{v+7}\right)v'=&\frac15\\
  \frac{v'}{v-7}-\frac{v'}{v+7}=&\frac{14}{5}\\
  \int\frac{v'}{v-7}\D t-\int\frac{v'}{v+7}\D t=&\int\frac{14}{5}\D t\\
  \int\frac{1}{v-7}\D(v-7)-\int\frac{1}{v+7}\D(v+7)=&\frac{14t}{5}+c\\
  \ln|v-7|-\ln|v+7|=&\frac{14t}{5}+c\\
  \ln\left(\left|\frac{v-7}{v+7}\right|\right)=&\frac{14t}{5}+c\\
  \left|\frac{v-7}{v+7}\right|=&ce^{\frac{14t}{5}}
\end{aligned}  
\]

Since $v(0)=v_0$, 
\[c=\left|\frac{v_0-7}{v_0+7}\right|\]

If $v_0\le -7$, then $v'(0)>0$ and $v(t)$ will be increasing until $v'=0$. So $v_0\le v(t)\le-7$. Then
\[\left|\frac{v-7}{v+7}\right|=\frac{v-7}{v+7}\]
and 
\[c=\frac{v_0-7}{v_0+7}.\]
So 
\[\frac{v-7}{v+7}=e^{\frac{14t}{15}}\cdot \frac{v_0-7}{v_0+7}.\]

If $v_0\ge -7$, then $v'(0)<0$ and $v(t)$ will be decreasing until $v'=0$. So $-7\le v(t)\le v_0$. Then
\[\left|\frac{v-7}{v+7}\right|=-\frac{v-7}{v+7}\]
and 
\[c=-\frac{v_0-7}{v_0+7}.\]
Again
\[\frac{v-7}{v+7}=e^{\frac{14t}{15}}\cdot \frac{v_0-7}{v_0+7}.\]

Solving for $v$ yields
\[v=-7\frac{(v_0-7)e^{\frac{14t}{5}}+v_0+7}{(v_0-7)e^{\frac{14t}{5}}-v_0-7}.\]

Since $v_0\le 0$, $v$ is defined and negative for all $t>0$. The terminal velocity is
\[\lim_{t\to\infty} v(t)=-7~\textup{ m/s},\]
independent of $v_0$.
\end{solution}

In examples above, we have been using international metric system for force. 
There are other metric systems (see \href{https://en.wikipedia.org/wiki/Newton\_(unit)}{Wiki page on Newton\_(unit)} for details). 
One of them is the British metric system which uses \emph{foot} (lb) for \emph{length}, \emph{pound-force} (lb) for \emph{force}, and \emph{slug} (sl) for \emph{mass}. 

\begin{exercise}
  A 320-lb object is given an initial upward velocity of 20 ft/s near the surface of Earth. The atmosphere resists the motion with a force of 2 lb for each ft/s of speed. Assuming that the only other force acting on the object is the gravity $g=32~\textup{ ft}/\textup{ s}^2$.
  
  Find its velocity $v$ as a function of $t$, and find its terminal velocity.
\end{exercise}
\begin{exersol}
  Note that here 320-lb means the gravitational force is 320 lb.
  Since $mg=320$ and $g=32$, $m=320/32=10~ \textup{ lb-s}^2/\textup{ ft}$. 
  Suppose the velocity function is $v=v(t)$ ft/s after $t$ 
  Then the atmospheric resistance is $-2v ~ \textup{ lb}$.
  
  By Newton's second law of motion, 
  \[10v'=-320-2v,\]
  equivalently,
  \[5v'=-160-v,\]
  Solving the equation by a linear substitution $y=v+1600$ yields
  \[
    \begin{aligned}
      5v'=&-160-v\\
      5y'=&-y\\
      \frac{y'}{y}=&-5\\
      \int \frac{y'}{y} \D t=&\int -5\D t\\
      \ln|y|=&-5t+c\\
      |y|=&ce^{-5t}\\
      |v+160|=&ce^{-5t}.
    \end{aligned}
    \]
Since $v(0)=20$ and $v'(0)<0$, the velocity $v$ will decrease. Note that $v'=0$ when $v=-160$. Then $-160\le v(t)\le 20$, and $|v+160|=v+160$.
Therefore, 
\[v=-160+ce^{-5t}.\] 
Since the initial velocity is $v(0)=20$, the constant $c$ satisfies the equation
\[20=-160+c.\]
Thus, $c=180$ and the velocity function is 
\[v=-160+180e^{-t/10}~\textup{ ft}/\textup{ s}.\]
  
The terminal velocity is
\[\lim\limits_{t\to\infty}v(t)=\lim\limits_{t\to\infty}-160+180e^{-t/10}=-160~\textup{ ft}/\textup{ s}.\]
\end{exersol}


\subsection{Orthogonal Trajectories*}

We've seen that general solutions to first order differential equations are in the form
\[F(x, y, c)=0,\]
where $c$ is a constant taking varies real values. The graph of those solutions are known as one-parameter families of curves. We will simply call them families of curves.

Two curves $C_1$ and $C_2$ are said to be \dfn{orthogonal} at a point of intersection $(x_0,y_0)$ if they have perpendicular tangents at $(x_0,y_0)$. 

A curve is said to be an \dfn{orthogonal trajectory} of a given family of curves if it is orthogonal to every curve in the family.

Suppose $F(x, y, c)=0$ is a family of integral curves of the differential equation
\[y'=-f(x,y).\]
From the definition, an orthogonal trajectory of $F(x, y, c)=0$ has perpendicular tangents. So slopes of tangent lines of an orthogonal trajectory are determined by 
\[y'=\frac{1}{f(x,y)}.\]

Therefore, families of integral curves of the differential equations $y'=f(x, y)$ and $y'=-\frac{1}{f(x, y)}$ are orthogonal trajectories to each other.

That suggests a method for finding orthogonal trajectories of a family of curves:
\begin{steps}
  \item Find a differential equation $y'=f(x,y)$ without $c$ for the given family $F(x, y, c)$.
  \item Solve the differential equation $y'=-\dfrac{1}{f(x,y)}$ to find the family of orthogonal trajectories.
\end{steps}

\begin{example}
  Find the family of orthogonal trajectories to the family of curves defined by
  \[y=cx^2, \quad c\ne0.\]
\end{example}
\begin{solution}
  Taking derivative implies
  \[y'=2cx.\]
  Since $y=cx^2$, solving for $c$ yields $c=\frac{y}{x^2}$. 
  So the above differential equation can be re-written as
  \[y'=\frac{2y}{x}.\]
  Therefore, orthogonal trajectories satisfy the equation
  \[y'=-\frac{x}{2y}.\]
  Solving the equation yields that the ellipse defined by
  \[y^2+\frac{1}{2}x^2=c,\]
  is a family of orthogonal trajectories.
\end{solution}

\begin{exercise}
  Find the orthogonal trajectories of the family of hyperbolas
\[xy=c \quad c\ne0.\]
\end{exercise}
\begin{exersol}
Differentiating the equation implicitly with respect to $x$ yields
\[y+xy'=0.\]
So
\[y'=-\frac{y}{x}.\]
Thus, the integral curves of
\[y'=\frac{x}{y}\]
are orthogonal trajectories of the given family. Separating variables and solving equations yields
\[
\begin{aligned}
  y'=&\frac{x}{y}\\
  y'y=&x\\
  \int y'y\D x=&\int x\D x\\
  y^2=&x^2+k\\
  y^2-x^2=&k.
\end{aligned}  
\]
So an orthogonal trajectory is either hyperbola (if $k \ne0$), or the union of the lines $y=x$ and $y=-x$ (if $k=0$).
\end{exersol}

\begin{subappendices}
\section{Integrating factor methods for linear first order equations revisited}
\subsection{The integrating factor method for linear first order equations}

Consider the differential equation
$$\dfrac{\mathrm{d}y}{\mathrm{d}x}+p(x)y=q(x).$$
Recall an integrating factor for this equation is
$$r(x)=\int e^{\int p(x)\mathrm{d}x}\mathrm{d}x.$$
Multiplying the equation by $r(x)$ yields
$$
\begin{aligned}
  r(x)\dfrac{\mathrm{d}y}{\mathrm{d}x}+r(x)p(x)y=&r(x)q(x)\\
  \dfrac{\mathrm{d}}{\mathrm{d}x}\left(r(x)y\right)=&r(x)q(x)\\
  r(x)y=&\int r(x)q(x)\mathrm{d} x\\
  y=&\frac{\int r(x)q(x)\mathrm{d} x}{r(x)}\\
  y=&\frac{\int(q(x)e^{\int p(x)\mathrm{d}x})\mathrm{d} x}{e^{\int p(x)\mathrm{d}x}}.
\end{aligned}
$$

\subsection{Linear first order equations viewed as non-exact equations}

This integrating method can be considered as a special case of the integrating method discussed today.
Note that the differential equation can be rewritten as
$$(p(x)y-q(x))\mathrm{d} x + \mathrm{d} y=0.$$
Let $M(x, y)=p(x)y-q(x)$ and $N(x, y)=1$. Then
$$\dfrac{\partial M}{\partial y}=p(x)\quad \textup{ and}\quad \dfrac{\partial N}{\partial x}=0.$$
Therefore,
$$\dfrac{\dfrac{\partial M}{\partial y}-\dfrac{\partial N}{\partial x}}{N}=p(x).$$
From the proposition discussed today, an integrating factor for the equation $(p(x)y-q(x))\mathrm{d} x + \mathrm{d} y$ is
$$\mu(x)=e^{\int p(x)\mathrm{d} x}.$$

From the above calculation, you may notice that $r(x)=\mu(x)$.

Now, to solve the equation $(p(x)y-q(x))\mathrm{d} x + \mathrm{d} y=0$, multiplying both sides by $\mu(x)$ yields an exact equation
$$e^{\int p(x)\mathrm{d}x}(p(x)y-q(x))\mathrm{d} x + e^{\int p(x)\mathrm{d}x} \mathrm{d} y=0.$$
The expected solution is an implicitly function defined by $F(x, y)=c$, that is $F$ is a function such that
$$\frac{\partial  F}{\partial x}\mathrm{d} x + \frac{\partial  F}{\partial y}\mathrm{d} y = \left[e^{\int p(x)\mathrm{d}x}(p(x)y-q(x))\right]\mathrm{d} x + e^{\int p(x)\mathrm{d}x} \mathrm{d} y.$$
To find $F$, we take the following steps:

1. Integrate $e^{\int p(x)\mathrm{d}x} \mathrm{d} y$. Denote the resulting function by $F(x, y)$. Then
    $$F(x, y)=ye^{\int p(x)\mathrm{d}x} + g(x).$$ The reason to add a $g(x)$ is because $\frac{\partial}{\partial y}g(x)=0$.

2. Solve for $g$ from the equation
    $$\frac{\partial F}{\partial x}=e^{\int p(x)\mathrm{d}x}(p(x)y-q(x)).$$
    Note that
    $$
    \begin{aligned}
    \frac{\partial F}{\partial x}=&y\frac{\partial}{\partial x}e^{\int p(x)\mathrm{d} x}+\frac{\partial}{\partial x}g(x)\\
    =&yp(x)e^{\int p(x)\mathrm{d} x}+\frac{\mathrm{d}}{\mathrm{d} x}g(x).
    \end{aligned}
    $$
  Then the function $g(x)$ satisfies the following equation
  $$yp(x)e^{\int p(x)\mathrm{d} x}+\frac{\mathrm{d}}{\mathrm{d} x}g(x)=e^{\int p(x)\mathrm{d}x}(p(x)y-q(x)),$$
  or equivalently,
  $$\frac{\mathrm{d}}{\mathrm{d} x}g(x)=-q(x)e^{\int p(x)\mathrm{d}x}.$$
  Therefore,
  $$g(x)=\int\left(-q(x)e^{\int p(x)\mathrm{d}x}\right)\mathrm{d}x,$$
  and
  $$F(x, y)=ye^{\int p(x)\mathrm{d}x}-\int\left(q(x)e^{\int p(x)\mathrm{d}x}\right)\mathrm{d}x.$$
  Set $F(x, y)=0$ and solve for $y$, we find a solution
  $$y=\dfrac{\int\left(q(x)e^{\int p(x)\mathrm{d}x}\right)\mathrm{d}x}{e^{\int p(x)\mathrm{d}x}},$$
  which is the seem as the solution obtained using the first method.

\begin{remark}
Instead of treating the whole equation as an non-exact equation and solve it, one can also find an integrating factor $\mu(x)$ of $yq(x)\mathrm{d} x + \mathrm{d} y$ and find $F$ so that
$$\frac{\partial F}{\partial x}\mathrm{d} x + \frac{\partial F}{\partial y}\mathrm{d} yq(x)\mathrm{d} x + \mathrm{d} y.$$
Then the equation $(yq(x)-q(x))\mathrm{d} x + \mathrm{d} y=0$ can be expressed as
$$\mathrm{d} F(x, y)=\mu(x)q(x)\mathrm{d} x.$$
Then integrating both sides yields the solution.
\end{remark}

\end{subappendices}